\newpage % Rozdziały zaczynamy od nowej strony.
\section{Praefatio}
\lipsum[1] \cite{goossens93}
\begin{figure}[!h]
	% Znacznik \caption oprócz podpisu służy również do wygenerowania numeru obrazka;
	\caption{Tradycyjne godło Politechniki Warszawskiej}
	% dlatego zawsze pamiętaj używać najpierw \caption, a potem \label
    \label{fig:tradycyjne-logo-pw}
    % Zamiast width można też użyć height, etc. 
    \centering \includegraphics[width=0.5\linewidth]{logopw.png}
\end{figure}
\lipsum[2-3]
\begin{figure}[!h]
	\caption{Współczesne logo Politechniki Warszawskiej}
	\label{fig:nowe-logo-pw}
	\centering \includegraphics[width=0.5\linewidth]{logopw2.png}
\end{figure}
\lipsum[4-6] Reference to image \ref{fig:nowe-logo-pw}. 
