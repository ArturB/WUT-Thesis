%%%%%%%%%%%%%%%%%%%%%%%%%%%%%%%%%%%%%%%%%%%%%%%%%%%%%%%
%% Engineer & Master Thesis, LaTeX Template          %%
%% Copyleft by Piotr Wozniak & Artur M. Brodzki      %%
%% Faculty of Electronics and Information Technology %%
%% Warsaw University of Technology, Warsaw, 2019     %%
%%%%%%%%%%%%%%%%%%%%%%%%%%%%%%%%%%%%%%%%%%%%%%%%%%%%%%%

\documentclass[
	left=2.5cm,       % Sadly, generic margin parameter
	right=2.5cm,      % doesnt't work, as it is
	top=2.5cm,        % superseded by more specific
	bottom=3cm,       % left...bottom parameters.
	bindingoffset=6mm % Optional binding offset. I advise to comment it, until you will be ready to print and bind your thesis. 
]{eiti/eiti-thesis}

\usepackage[polish]{babel}

%----------------------------------
% Alternatywne czcionki
%----------------------------------
%\usepackage{fontspec} % custom fonts
%\setmainfont{Adagio Slab}
%\setromanfont{times.ttf}
%\setsansfont{arial.ttf}

%----------------------------------
% Twierdzenia i definicje;
% tutaj ew. tłumaczymy te terminy
% na inne języki
%----------------------------------
\newtheorem{theorem}{Twierdzenie}
\newtheorem{lemma}{Lemat}
\newtheorem{corollary}{Wniosek}
\newtheorem{definition}{Definicja}
\newtheorem{axiom}{Aksjomat}
\newtheorem{assumption}{Założenie}

\begin{document}

%--------------------------------
% Strona tytułowa
%--------------------------------
\MasterThesis % dla pracy inżynierskiej mamy \EngineerThesis
\instytut{XXXXXX}
\kierunek{XXXXXX}
\specjalnosc{XXXXXX}
\title{
	Niepotrzebnie długi i skomplikowany tytuł pracy \\ 
	trudny do przeczytania, zrozumienia i wymówienia}
\author{\{Imię i Nazwisko\}}
\album{XXXXXX}
\promotor{XXXXXX}
\date{\the\year}
\maketitle

%--------------------------------
% Streszczenie po polsku
%--------------------------------
\streszczenie \lipsum[1-3]
\slowakluczowe XXX, XXX, XXX
\newpage

%--------------------------------
% Streszczenie po angielsku
%--------------------------------
\abstract \kant[1-3]
\keywords XXX, XXX, XXX
\newpage

%--------------------------------
% Oświadczenie o autorstwie
%--------------------------------
\makeauthorship
\blankpage

%--------------------------------
% Spis treści
%--------------------------------
\thispagestyle{empty}
\tableofcontents
\blankpage

%--------------------------------
% Rozdziały
%--------------------------------

\section{Wstęp}
\lipsum[1] \cite{greenwade93}
\begin{figure}[!h]
	\label{fig:anzelm}
	\centering \includegraphics[width=0.5\linewidth]{img/logopw.png}
	\caption{Tradycyjne godło Politechniki Warszawskiej.}
\end{figure}
\lipsum[2-10]
         % W długich pracach
\clearpage % Rozdziały zaczynamy od nowej strony.
\section{De Finibus Bonorum et Malorum}

% Równanie typu 'inline':
\lipsum[2] $F = m \cdot a$ lorem ipsum dolor sit amet.
% Równanie bez numeru
% align oznacza wyrównanie kolejnych wierszy do '&'
% '&' służy tylko do wyrównania i nie jest renderowany
\begin{align*}
    E & = mc^2 \\
    y & = ax^2 + bx + c
\end{align*}

\lipsum[3]
% Równanie numerowane: macierze
\begin{align}
    \begin{bmatrix}
        1 & 0 & 0 \\
        0 & 2 & 0 \\
        0 & 0 & 3
    \end{bmatrix} \cdot
    \begin{bmatrix}
        4 \\
        5 \\
        6
    \end{bmatrix} =
    \begin{bmatrix}
        4  \\
        10 \\
        18
    \end{bmatrix}
\end{align}

% Cytaty dla zapełnienia bibliografii
\lipsum[4] Lorem ipsum dolor sit amet, consectetur adipiscing elit, sed do eiusmod tempor incididunt ut labore et dolore magna aliqua \cite{szczypiorski2015}, \cite{duqu2011}, \cite{shs2015}, \cite{wozniak2018}, \cite{dcp19}.

% Podrozdział pierwszego poziomu
\subsection{Critique of Pure Reason}
\kant[1]

% Tabela wielostronicowa, 4 kolumny
% Kolumny typu m{} oznaczają kolumny o stałej szerokości z zawijaniem wierszy
% Wyrównywane są domyślnie do lewej; aby ustawić inne wyrównanie,
% stosujemy \multicolumn{1} tak jak poniżej
\begin{longtable}{| c | m{0.58\linewidth} | r | m{0.1\linewidth} |}
    \caption{Tabela wielostronicowa.}
    \label{table:koszty} \\

    \hline
    % Nagłówek tabeli wyrównujemy do środka
    Lp & \multicolumn{1}{c|}{Treść} & \multicolumn{1}{c|}{Kwota} & \multicolumn{1}{m{0.1\linewidth}|}{Wariant opłaty} \\ \hline\hline \endfirsthead \endfoot
    \hline \endlastfoot

    1  & Lorem ipsum dolor sit amet, consectetur adipiscing elit, sed do eiusmod tempor incididunt ut labore et dolore magna aliqua. & 111 111,11 zł & \multicolumn{1}{c|}{WAR1} \\ \hline
    2  & Lorem ipsum dolor sit amet, consectetur adipiscing elit, sed do eiusmod tempor incididunt ut labore et dolore magna aliqua. & 22 222,22 zł & \multicolumn{1}{c|}{WAR1} \\ \hline
    3  & Lorem ipsum dolor sit amet, consectetur adipiscing elit, sed do eiusmod tempor incididunt ut labore et dolore magna aliqua. & 33 333,33 zł & \multicolumn{1}{c|}{WAR1} \\ \hline
    4  & Lorem ipsum dolor sit amet, consectetur adipiscing elit, sed do eiusmod tempor incididunt ut labore et dolore magna aliqua. & 444 444,44 zł & \multicolumn{1}{c|}{WAR1} \\ \hline
    5  & Lorem ipsum dolor sit amet, consectetur adipiscing elit, sed do eiusmod tempor incididunt ut labore et dolore magna aliqua. & 55 555,55 zł & \multicolumn{1}{c|}{WAR1} \\ \hline
    6  & Lorem ipsum dolor sit amet, consectetur adipiscing elit, sed do eiusmod tempor incididunt ut labore et dolore magna aliqua. & 66 666,66 zł & \multicolumn{1}{c|}{WAR1} \\ \hline
    7  & Lorem ipsum dolor sit amet, consectetur adipiscing elit, sed do eiusmod tempor incididunt ut labore et dolore magna aliqua. & 777 777,77 zł & \multicolumn{1}{c|}{WAR1} \\ \hline
    8  & Lorem ipsum dolor sit amet, consectetur adipiscing elit, sed do eiusmod tempor incididunt ut labore et dolore magna aliqua. & 8 888,88 zł & \multicolumn{1}{c|}{WAR1} \\ \hline
    9  & Lorem ipsum dolor sit amet, consectetur adipiscing elit, sed do eiusmod tempor incididunt ut labore et dolore magna aliqua. & 999 999,99 zł & \multicolumn{1}{c|}{WAR1} \\ \hline
    10 & Lorem ipsum dolor sit amet, consectetur adipiscing elit, sed do eiusmod tempor incididunt ut labore et dolore magna aliqua. & 111 111,11 zł & \multicolumn{1}{c|}{WAR2} \\ \hline
    11 & Lorem ipsum dolor sit amet, consectetur adipiscing elit, sed do eiusmod tempor incididunt ut labore et dolore magna aliqua. & 22 222,22 zł & \multicolumn{1}{c|}{WAR2} \\ \hline
    12 & Lorem ipsum dolor sit amet, consectetur adipiscing elit, sed do eiusmod tempor incididunt ut labore et dolore magna aliqua. & 33 333,33 zł & \multicolumn{1}{c|}{WAR2} \\ \hline
    13 & Lorem ipsum dolor sit amet, consectetur adipiscing elit, sed do eiusmod tempor incididunt ut labore et dolore magna aliqua. & 444 444,44 zł & \multicolumn{1}{c|}{WAR2} \\ \hline
    14 & Lorem ipsum dolor sit amet, consectetur adipiscing elit, sed do eiusmod tempor incididunt ut labore et dolore magna aliqua. & 55 555,55 zł & \multicolumn{1}{c|}{WAR2} \\ \hline
    15 & Lorem ipsum dolor sit amet, consectetur adipiscing elit, sed do eiusmod tempor incididunt ut labore et dolore magna aliqua. & 66 666,66 zł & \multicolumn{1}{c|}{WAR2} \\ \hline
       & \multicolumn{1}{r|}{\textbf{Suma:}} & \textbf{7 777 777,77 zł} &
\end{longtable}

\kant[2]

% Nagłówki kolejnych poziomów, dla zapełnienia spisu treści
\subsection{Caegorical Imperative} % 2.2
\subsubsection{Deontological Ethics} % 2.2.1
\kant[2]
\subsubsection{Consequentialism -- the Ideal of practical reason} % 2.2.2
\kant[3]
\subsection{G\"odel's ontological proof} % 2.3
\kant[9] Lorem ipsum dolor sit amet, consectetur adipiscing elit \cite{benzmuller2014}, \cite{goedel95}, \cite{wang97}, \cite{koons2005}.

% Twierdzenia i dowody
% Założenie
\begin{assumption} \label{ass:1}
    $ [\![ \ \phi \ ]\!] \Longrightarrow [\![ \ P(\phi); \neg P(\phi) \ ]\!]$
\end{assumption}
% Aksjomat
\begin{axiom}[Dualność] \label{axiom:1}
    $\neg P(\phi) \Leftrightarrow P(\neg \phi)$, równoważnie $P(\phi) \Leftrightarrow \neg P(\neg \phi)$
\end{axiom}
\begin{axiom}[Całkowitość] \label{axiom:2}
    $ \left( P(\phi) \wedge \forall x: \phi(x) \Rightarrow \psi(x) \right) \Rightarrow P(\psi) $
\end{axiom}
\begin{axiom}[Absolutność] \label{axiom:3}
    $ P(\phi) \Rightarrow \Box P(\phi) $
\end{axiom}
% Definicja
\begin{definition} \label{def:1}
    $ G(x) \Leftrightarrow \forall \phi: \left( P(\phi) \Rightarrow \phi(x) \right) $
\end{definition}
\begin{definition} \label{def:2}
    $ \phi \ ess \ x \Leftrightarrow \phi(x) \wedge \forall \psi \left( \psi(x) \Rightarrow \Box \forall y \left( \phi(y) \Rightarrow \psi(y) \right) \right)  $
\end{definition}
\begin{axiom} \label{axiom:4}
    P(G)
\end{axiom}
% Lemat
\begin{lemma} \label{lemma:1}
    $ P(\phi) \Rightarrow \Diamond \exists x : \phi(x) $
\end{lemma}
\begin{proof}
    Dowód pomijamy, bo jest trywialny :)
\end{proof}
\begin{lemma} \label{lemma:2}
    $ \Diamond \exists x : G(x) $
\end{lemma}
\begin{proof}
    Natychmiastowy wniosek z aksjomatu \ref{axiom:4} i lematu \ref{lemma:1}.
\end{proof}
\begin{lemma} \label{lemma:3}
    $ G(x) \Rightarrow G \ ess \ x $
\end{lemma}
\begin{proof}
    Poprzez podstawienie do definicji \ref{def:2}.
\end{proof}
\begin{definition} \label{def:3}
    $ E(x) \Leftrightarrow \forall \phi \left( \phi \ ess \ x \Rightarrow \Box\ \exists x: \phi(x) \right) $
\end{definition}
\begin{axiom} \label{axiom:5}
    P(E)
\end{axiom}
% Twierdzenie
\begin{theorem}
    $ \Box\ \exists x : G(x) $
\end{theorem}
\begin{proof}
    Na podstawie definicji \ref{def:1}, lematu \ref{lemma:3} i aksjomatu \ref{axiom:5}.
\end{proof}
    % wygodnie jest trzymać
\newpage % Rozdziały zaczynamy od nowej strony.
\section{Code listings}
\lipsum[10]
\begin{lstlisting}[language=HTML]
<html>
  <head>
    <title>Hello world!</title>
  </head>
  <body>
    Hello
  </body>
</html>
\end{lstlisting}
\lipsum[11]
\begin{lstlisting}[language=C]
#include <stdio.h>
int main() {
  // printf() displays the string inside quotation
  printf("Hello world!");
  return 0;
}
\end{lstlisting}
\lipsum[12]
 % każdy rozdział w osobnym pliku. 

\section{Podsumowanie}   % Ale można też pisać w jednym. 
\lipsum[5-7]

%--------------------------------
% Literarura
%--------------------------------
\newpage
\bibliographystyle{ieeetr}
\bibliography{literatura}

%--------------------------------
% Spisy
%--------------------------------
\newpage
\listoffigures % Spis obrazków. 
\listoftables  % Spis tabel. 

\end{document}

